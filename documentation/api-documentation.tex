\documentclass[12pt,a4paper]{article}

% Packages
\usepackage[utf8]{inputenc}
\usepackage[T1]{fontenc}
\usepackage{graphicx}
\usepackage{booktabs}
\usepackage{longtable}
\usepackage{geometry}
\usepackage{hyperref}
\usepackage{xcolor}
\usepackage{listings}
\usepackage{fancyhdr}
\usepackage{titlesec}
\usepackage{tocloft}
\usepackage{tabularx}

% Page geometry
\geometry{margin=2.5cm}

% Colors
\definecolor{primaryblue}{RGB}{17, 138, 178}
\definecolor{darkgray}{RGB}{51, 51, 51}
\definecolor{lightgray}{RGB}{245, 245, 245}
\definecolor{codegreen}{RGB}{0, 128, 0}
\definecolor{codepurple}{RGB}{128, 0, 128}
\definecolor{backcolour}{RGB}{250, 250, 250}

% Hyperref setup
\hypersetup{
    colorlinks=true,
    linkcolor=primaryblue,
    filecolor=primaryblue,
    urlcolor=primaryblue,
}

% Listings setup for JSON
\lstdefinelanguage{json}{
    basicstyle=\ttfamily\small,
    numbers=none,
    showstringspaces=false,
    breaklines=true,
    frame=single,
    backgroundcolor=\color{backcolour},
    literate=
        *{:}{{{\color{codepurple}{:}}}}{1}
        {,}{{{\color{codepurple}{,}}}}{1}
        {\{}{{{\color{codepurple}{\{}}}}{1}
        {\}}{{{\color{codepurple}{\}}}}}{1}
        {[}{{{\color{codepurple}{[}}}}{1}
        {]}{{{\color{codepurple}{]}}}}{1},
}

% Header and footer
\pagestyle{fancy}
\fancyhf{}
\fancyhead[L]{\textbf{Energy Monitoring System}}
\fancyhead[R]{REST API Documentation}
\fancyfoot[C]{\thepage}
\renewcommand{\headrulewidth}{0.4pt}

% Title formatting
\titleformat{\section}{\Large\bfseries\color{primaryblue}}{\thesection}{1em}{}
\titleformat{\subsection}{\large\bfseries\color{darkgray}}{\thesubsection}{1em}{}
\titleformat{\subsubsection}{\normalsize\bfseries\color{darkgray}}{\thesubsubsection}{1em}{}

% HTTP Method colors
\newcommand{\httpget}{\colorbox{green!30}{\textbf{GET}}}
\newcommand{\httppost}{\colorbox{blue!30}{\textbf{POST}}}
\newcommand{\httpput}{\colorbox{yellow!30}{\textbf{PUT}}}
\newcommand{\httpdelete}{\colorbox{red!30}{\textbf{DELETE}}}

% Document info
\title{
    \vspace{-2cm}
    \textbf{\Huge REST API Documentation}\\[0.5cm]
    \Large Energy Monitoring Dashboard System\\[0.3cm]
    \large Technical Documentation
}
\author{Solutions Architect Team\\Ravelware}
\date{Version 1.0 -- January 16, 2026}

\begin{document}

\maketitle
\thispagestyle{empty}

\vspace{1cm}

\begin{abstract}
This document provides comprehensive documentation for the REST API endpoints of the Energy Monitoring Dashboard System. All endpoints are prefixed with \texttt{/api/v1} and return JSON responses. The API supports real-time panel monitoring, historical data retrieval, and MQTT client management.
\end{abstract}

\vspace{1cm}

\tableofcontents
\newpage

% ============================================
\section{Document Information}
% ============================================

\begin{table}[h]
\centering
\begin{tabular}{@{}ll@{}}
\toprule
\textbf{Property} & \textbf{Value} \\
\midrule
Document Title & REST API Documentation \\
Version & 1.0 \\
Date & January 16, 2026 \\
Author & Solutions Architect Team \\
Base URL & \texttt{http://localhost:3000/api/v1} \\
Content-Type & application/json \\
\bottomrule
\end{tabular}
\end{table}

% ============================================
\section{API Overview}
% ============================================

% taruh diagram di sini
\begin{figure}[h]
\centering
\includegraphics[width=0.95\textwidth]{plantuml/api-endpoints-diagram.png}
\caption{API Endpoints Sequence Diagram}
\label{fig:api-sequence}
\end{figure}

\subsection{Base URL}
\begin{verbatim}
http://localhost:3000/api/v1
\end{verbatim}

\subsection{Authentication}
Currently, the API does not require authentication. Future versions may implement JWT-based authentication.

\subsection{Response Format}
All API responses follow this standard format:

\begin{lstlisting}[language=json]
{
  "status": "OK" | "ERROR",
  "message": "Optional message",
  "data": { ... }
}
\end{lstlisting}

\subsection{HTTP Status Codes}

\begin{table}[h]
\centering
\begin{tabular}{@{}lll@{}}
\toprule
\textbf{Code} & \textbf{Status} & \textbf{Description} \\
\midrule
200 & OK & Request successful \\
400 & Bad Request & Invalid parameters \\
404 & Not Found & Resource not found \\
500 & Internal Server Error & Server error \\
\bottomrule
\end{tabular}
\end{table}

% ============================================
\section{Endpoints}
% ============================================

% --------------------------------------------
\subsection{\httpget{} /panels/realtime}
% --------------------------------------------

\textbf{Description:} Get real-time data for all panels including current readings and status.

\subsubsection{Request}
\begin{verbatim}
GET /api/v1/panels/realtime
\end{verbatim}

\subsubsection{Parameters}
None

\subsubsection{Response}
\begin{lstlisting}[language=json]
{
  "status": "OK",
  "data": {
    "panels": [
      {
        "pmCode": "PANEL_LANTAI_1",
        "location": "Lantai 1 - Main",
        "floor": 1,
        "panelStatus": "ONLINE",
        "lastUpdateRelative": "2s ago",
        "v": [224.7, 224.7, 223.5, 149.8],
        "i": [0.8, 0.99, 0.58, 0.03],
        "kw": "0.36",
        "kVA": "0.46",
        "kWh": "150.07",
        "pf": 0.86,
        "vunbal": 0.01,
        "iunbal": 0.047,
        "time": "2026-01-16 11:30:05"
      }
    ],
    "timestamp": "2026-01-16 11:30:10"
  }
}
\end{lstlisting}

\subsubsection{Response Fields}

\begin{table}[h]
\centering
\small
\begin{tabular}{@{}lll@{}}
\toprule
\textbf{Field} & \textbf{Type} & \textbf{Description} \\
\midrule
pmCode & string & Panel code identifier \\
location & string & Physical location \\
floor & integer & Floor number \\
panelStatus & string & ONLINE or OFFLINE \\
lastUpdateRelative & string & Relative time since last update \\
v & array[4] & Voltage [R, S, T, N] in Volts \\
i & array[4] & Current [R, S, T, N] in Amperes \\
kw & string & Active power in kW \\
kVA & string & Apparent power in kVA \\
kWh & string & Cumulative energy in kWh \\
pf & number & Power factor (0-1) \\
vunbal & number & Voltage unbalance ratio \\
iunbal & number & Current unbalance ratio \\
time & string & Timestamp (YYYY-MM-DD HH:mm:ss) \\
\bottomrule
\end{tabular}
\end{table}

% --------------------------------------------
\subsection{\httpget{} /panels/usage/today/:panelCode}
% --------------------------------------------

\textbf{Description:} Get today's energy usage and cost for a specific panel.

\subsubsection{Request}
\begin{verbatim}
GET /api/v1/panels/usage/today/PANEL_LANTAI_1
\end{verbatim}

\subsubsection{Path Parameters}

\begin{table}[h]
\centering
\begin{tabular}{@{}llll@{}}
\toprule
\textbf{Parameter} & \textbf{Type} & \textbf{Required} & \textbf{Description} \\
\midrule
panelCode & string & Yes & Panel identifier (e.g., PANEL\_LANTAI\_1) \\
\bottomrule
\end{tabular}
\end{table}

\subsubsection{Response}
\begin{lstlisting}[language=json]
{
  "status": "OK",
  "data": {
    "panelCode": "PANEL_LANTAI_1",
    "date": "2026-01-16",
    "todayUsageKWh": 18.97,
    "todayCost": 28455,
    "currency": "IDR",
    "currentKWh": 150.07,
    "midnightKWh": 131.1
  }
}
\end{lstlisting}

\subsubsection{Calculation Logic}
\begin{verbatim}
todayUsageKWh = currentKWh - midnightKWh
todayCost = todayUsageKWh * COST_PER_KWH (Rp 1.500/kWh)
\end{verbatim}

% --------------------------------------------
\subsection{\httpget{} /panels/history/:panelCode}
% --------------------------------------------

\textbf{Description:} Get historical energy data for chart visualization.

\subsubsection{Request}
\begin{verbatim}
GET /api/v1/panels/history/PANEL_LANTAI_1?range=24h
\end{verbatim}

\subsubsection{Parameters}

\begin{table}[h]
\centering
\begin{tabular}{@{}lllll@{}}
\toprule
\textbf{Parameter} & \textbf{Type} & \textbf{In} & \textbf{Required} & \textbf{Description} \\
\midrule
panelCode & string & path & Yes & Panel identifier \\
range & string & query & No & Time range (default: 24h) \\
\bottomrule
\end{tabular}
\end{table}

\subsubsection{Valid Range Values}

\begin{table}[h]
\centering
\begin{tabular}{@{}lll@{}}
\toprule
\textbf{Range} & \textbf{Aggregation} & \textbf{Description} \\
\midrule
1h & 1 hour & Last 1 hour \\
6h & 1 hour & Last 6 hours \\
12h & 1 hour & Last 12 hours \\
24h & 1 hour & Last 24 hours \\
7d & 6 hours & Last 7 days \\
30d & 1 day & Last 30 days \\
1y & 30 days & Last 1 year \\
\bottomrule
\end{tabular}
\end{table}

\subsubsection{Response}
\begin{lstlisting}[language=json]
{
  "status": "OK",
  "message": "",
  "data": {
    "pmCode": "PANEL_LANTAI_1",
    "year": "2026",
    "month": "01",
    "date": "16",
    "range": "24h",
    "energy": [131.179, 131.523, 131.87, 132.219, ...],
    "cost": [196768, 197284, 197805, 198329, ...],
    "dataPoints": 24
  }
}
\end{lstlisting}

% --------------------------------------------
\subsection{\httpget{} /panels/usage/monthly/:year/:month}
% --------------------------------------------

\textbf{Description:} Get monthly energy usage summary for all panels.

\subsubsection{Request}
\begin{verbatim}
GET /api/v1/panels/usage/monthly/2026/01
\end{verbatim}

\subsubsection{Path Parameters}

\begin{table}[h]
\centering
\begin{tabular}{@{}llll@{}}
\toprule
\textbf{Parameter} & \textbf{Type} & \textbf{Required} & \textbf{Description} \\
\midrule
year & string & Yes & Year (e.g., 2026) \\
month & string & Yes & Month 01-12 (e.g., 01) \\
\bottomrule
\end{tabular}
\end{table}

\subsubsection{Response}
\begin{lstlisting}[language=json]
{
  "status": "OK",
  "data": {
    "year": 2026,
    "month": 1,
    "panels": [
      {
        "panelCode": "PANEL_LANTAI_1",
        "totalKWh": 450.5,
        "totalCost": 675750
      },
      {
        "panelCode": "PANEL_LANTAI_2",
        "totalKWh": 380.2,
        "totalCost": 570300
      }
    ],
    "buildingTotal": {
      "totalKWh": 1200.75,
      "totalCost": 1801125
    }
  }
}
\end{lstlisting}

% --------------------------------------------
\subsection{\httppost{} /mqtt/start}
% --------------------------------------------

\textbf{Description:} Start the MQTT client listener to receive sensor data.

\subsubsection{Request}
\begin{verbatim}
POST /api/mqtt/start
\end{verbatim}

\subsubsection{Request Body}
None

\subsubsection{Response}
\begin{lstlisting}[language=json]
{
  "status": "OK",
  "message": "MQTT client started",
  "connected": true
}
\end{lstlisting}

% --------------------------------------------
\subsection{\httpget{} /mqtt/start}
% --------------------------------------------

\textbf{Description:} Check MQTT client connection status.

\subsubsection{Request}
\begin{verbatim}
GET /api/mqtt/start
\end{verbatim}

\subsubsection{Response}
\begin{lstlisting}[language=json]
{
  "status": "OK",
  "connected": true,
  "initialized": true
}
\end{lstlisting}

% ============================================
\section{Error Responses}
% ============================================

\subsection{400 Bad Request}
\begin{lstlisting}[language=json]
{
  "status": "ERROR",
  "message": "Invalid range. Valid values: 1h, 6h, 12h, 24h, 7d, 30d, 1y"
}
\end{lstlisting}

\subsection{404 Not Found}
\begin{lstlisting}[language=json]
{
  "status": "ERROR",
  "message": "Panel not found"
}
\end{lstlisting}

\subsection{500 Internal Server Error}
\begin{lstlisting}[language=json]
{
  "status": "ERROR",
  "message": "Failed to fetch data",
  "error": "Database connection error"
}
\end{lstlisting}

% ============================================
\section{Glossary}
% ============================================

\begin{description}
    \item[kWh] Kilowatt-hour -- Unit of energy consumption
    \item[kW] Kilowatt -- Unit of power
    \item[kVA] Kilovolt-ampere -- Unit of apparent power
    \item[Power Factor] Ratio of real power to apparent power
    \item[MQTT] Message Queuing Telemetry Transport protocol
    \item[IDR] Indonesian Rupiah currency
\end{description}

\end{document}
